\documentclass{jsarticle}
\usepackage[dvipdfmx]{graphicx}
\title{8.工作の基礎}

\author{03-190697 高松周平}
\date{\today}
\begin{document}
\maketitle
\section{概要}
\subsection{実験の目的}
金属材料の加工と電子部品のはんだ付けなどの演習を行うことで、はんだ付けや金属加工機などの使い方を学ぶ。
\subsection{原理}
\subsubsection{二石トランジスタラジオ}
\begin{itemize}
\item 同調回路\\
コイルとコンデンサーを並列につなぐことで一定の周波数の波のみ通す回路である。その同調周波数は以下の式によって決まる。\\
$$
F = \frac{1}{2\pi\sqrt{LC}}\ \ \ L:コイルのインダクタンス、C:コンデンサの容量
$$
\item 高周波増幅回路\\
トランジスタとコイルを用いて高周波のみを増幅する回路
\item 検波回路\\
ダイオードと抵抗とコンデンサーを用いてAM波の音声信号部分のみを取り出す回路。
\item 低周波増幅回路\\
音声信号である低周波信号を増幅する回路。
\end{itemize}
\subsubsection{小型ジャッキ}
\begin{itemize}
\item バンドソー\\
素材を切断するための工作機械。切断刃を回転させ、それで素材を切断する。切断の仕組みにより加工精度はそこまで高くはないので大まかにきり、その後フライス盤やヤスリなどで目標の寸法にすることが多い。素材の硬さなどの特性などで刃の速度などを考えないと割れたりと危険である。
\item フライス盤\\
テーブルの上に固定した素材に回転している切リ刃を当てることで素材を削り目標の寸法に仕上げる。素材のテーブルはx,y,zの三軸に動くので様々な方向から削ることができる。また切リ刃には様々な種類があり、目標に合わせて付け替える。
\item ボール盤\\
回転するドリルを上下に動かすことができてテーブルの上に固定した素材に穴を開けることができる。
\item 旋盤\\
素材を固定することでその素材を高速で回転させることができるもの。それに様々な刃を当てることで円柱の径を整えたり、溝をほったりすることができる。非常に大きな摩擦熱が生じるので使えな材料も存在する。
\item タップ・ダイス\\
ねじ切りをするためのもの。
\end{itemize}
\subsection{手順}
\subsubsection{二石トランジスタラジオ}
まず、アンテナ(今回はビニール線を用いる)から電波を拾う。そして、リードインダクタとポリバルコンでできる同調回路によって、必要な周波数だけを選ぶ。そして、高周波増幅回路に入力して、検波回路で検波して音声を取り出し、低周波増幅回路でイヤホンからの音を出すようになっている。
\subsubsection{小型ジャッキ}
大きく分けて2つの部分に分かれる。ブロック部とジャッキ駆動部がある。ブロッグ部は真鍮の角材をバンドソーを用いて寸法の形に切断し、フライス盤でより精度高く整える。そしてボール盤を用いて穴をあけ、ねじ切り加工でこの穴をねじの形に加工する。次に、ジャッキ駆動部はジュラルミンの円柱を旋盤を用いて、径を整え、ボール盤でハンドル用の穴を開ける。また、ねじ切りを用いてネジの形に加工する。
\subsection{注意点}
\subsubsection{二石トランジスタラジオ}
はんだこてに電源が入ったまま放置するのは良くない
\subsubsection{小型ジャッキ}
手袋をして作業しない\\
\section{実験結果}
\subsubsection{二石トランジスタラジオ}
非常に良くできました
\subsubsection{小型ジャッキ}
非常に良くできました
\section{考察}
\subsection{二石トランジスタラジオ}
今回作ったラジオはとても簡易な精度の低いラジオであった。高性能なラジオとして一般的にスーパーヘテロダイン方式というものが存在する。今回の回路の高周波増幅のあとに、周波数変換とフィルターと中間周波増幅を入れる。発振回路によりもとの周波よりも445kHzだけ高い周波数を発生させ、もとの波とのうねりにより445kHzの中間波の信号を生成する。この波は他局の波と大きく異なるのでフィルターでこの周波数だけを拾おうとすると混信を減らして拾うことが可能となる。
\subsection{小型ジャッキ}
今回ジャッキ駆動部に用いた素材はジュラルミンである。ジュラルミンはCuを含むアルミ合金である。この素材の特性はもともと加工性が高く強度が低かったアルミに比べ強度が上がる代わりに加工性は低くなる。また鍛造することが可能なので様々な部品に用いられている。

\section{参考文献}
\begin{thebibliography}{99}
\bibitem{} 教科書
\bibitem{} 第2章 AM検波 http://www.rf-world.jp/dls/fujihira/pdf/Fujihira-radio-s02.pdf
\bibitem{} ラジオの動作原理 http://member.tokoha-u.ac.jp/~kdeguchi/hobby/radio/detection.html
\bibitem{} A2017(ジュラルミン)の成分、特性、機械的性質、強度、耐力について https://www.toishi.info/sozai/al/a2017.html
\end{thebibliography} 

\end{document}